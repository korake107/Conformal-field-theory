\documentclass[11pt]{article}
\usepackage[utf8]{inputenc}
\usepackage[margin=35truemm]{geometry} % 余白を35mmに設定

% --- 数学関連のパッケージ ---
\usepackage{amsmath}      % 様々な数式環境
\usepackage{amsthm}       % 定理環境
\usepackage{amsfonts}     % 数学用フォント
\usepackage{physics}      % 物理学の数式用マクロ (bra, ket, dvなど)
\usepackage{bm}           % 太字の数式 \bm{}

% --- 図や色、レイアウト関連のパッケージ ---
\usepackage[dvipdfmx]{graphicx, color} % 図の挿入や色の利用
\usepackage{tikz}         % 図形描画
\usetikzlibrary{intersections,calc,arrows.meta}
\usepackage{float}        % 図表の位置を調整 [H]など
\usepackage{siunitx}      % 単位をきれいに表示 \SI{100}{\kilo\gram}など
\usepackage{ascmac}       % itemboxなどの囲み枠

% --- その他 ---
\parindent = 0pt % 全体の段落開始時のインデントをなくす

% --- ハイパーリンク関連 (できるだけ最後に読み込む) ---
\usepackage[
    dvipdfmx,
    bookmarks=true,
    bookmarksnumbered=true,
    bookmarkstype=toc
]{hyperref}
\usepackage{pxjahyper} % hyperrefの日本語文字化け対策
\begin{document}

\title{共形場理論}
\author{小武 龍斗}
\date{\today}
\maketitle
\section{}
\section{共形変換}
まず共形変換の定義を与える。共形変換は、二つの多様体$M,N$の開集合$U,V$に対して、微分可能写像を$\varphi: U \to V$とする。
$g$と$g'$はそれぞれ$M$と$N$の計量テンソルである。
このとき引き戻しが$$\varphi^* g' = \Omega g$$を満たすとき、共形変換であると言う。ここで$\Omega$は$U$上のスカラー関数である。\\
このとき$\phi=x'$とすると$T_pM$の基底ベクトル$\pdv{x^\mu},\pdv{x^\nu}$を代入して$$g'\qty(\phi_*\qty(\pdv{x^\mu}),\phi_*\qty(\pdv{x^\nu})) = \Lambda g\qty(\pdv{x^\mu},\pdv{x^\nu})$$が成り立つ。
\begin{align}
    g'\qty(\pdv{x^\rho}{x^\mu}\pdv{x'^\rho},\pdv{x^\sigma}{x^\nu}\pdv{x'^\sigma}) &= \Lambda g\qty(\pdv{x^\mu},\pdv{x^\nu})\\
    g'_{\rho\sigma}\pdv{x^\rho}{x^\mu}\pdv{x^\sigma}{x^\nu} &= \Lambda g_{\mu\nu}
\end{align}
ここでは$M=M',g=g'$とすると次の関係が成り立つ。
\begin{align}
    \eta_{\rho \sigma}\pdv{x'^\rho}{x^\mu}\pdv{x'^\sigma}{x^\nu} &= \Lambda(x) \eta_{\mu\nu}\label{3}
\end{align}
次に線素を考える。線素は$ds^2 = \eta_{\mu\nu}dx^\mu dx^\nu$で与えられる。共形変換の下での線素は
\begin{align}
ds^2= \eta_{\rho \sigma}dx'^\rho dx'^\sigma
= \eta_{\rho \sigma}\pdv{x'^\rho}{x^\mu}\pdv{x'^\sigma}{x^\nu}dx^\mu dx^\nu
= \Lambda \eta_{\mu\nu}dx^\mu dx^\nu
\end{align}
つまり、共形変換とはローレンツ変換とスケール変換の組み合わせであることがわかる。
\subsection{共形群}
2つの共形変換の合成も共形変換であるので、共形変換全体は群を成す。この群を共形群と呼ぶ。$\Lambda(x)=1$ならポアンカレ群に対応する。
\subsection{無限小変換}
flat spaceにおける無限小共形変換を考える。$x^\mu \to x'^\mu = x^\mu + \epsilon^\mu(x) (\epsilon<<1)$とする。
これを(\ref{3})式に代入すると
\begin{align}
    \eta_{\rho \sigma}\pdv{x'^\rho}{x^\mu}\pdv{x'^\sigma}{x^\nu} &= \Lambda(x) \eta_{\mu\nu}\\
    \eta_{\rho \sigma}\qty(\delta^\rho_\mu + \pdv{\epsilon^\rho}{x^\mu})\qty(\delta^\sigma_\nu + \pdv{\epsilon^\sigma}{x^\nu}) &= \qty(1+K(x)) \eta_{\mu\nu}\\
    \partial_\mu \epsilon_\nu + \partial_\nu \epsilon_\mu &= K(x) \eta_{\mu\nu}\label{7}
\end{align}
この$K(x)$をどうにか表せないか考える。トレースをとる操作を行う。つまり両辺に$\eta^{\nu\mu}$
\begin{align}
    \partial_\mu \epsilon^\mu + \partial_\nu \epsilon^\nu &= K(x) \eta_{\mu\nu}\eta^{\nu\mu}\\
    K(x)=\frac{2\partial^\mu\epsilon_\mu}{d}\\
\end{align}
これを(\ref{7})式に代入すると
\begin{align}
    \partial_\mu \epsilon_\nu + \partial_\nu \epsilon_\mu &= \frac{2}{d}(\partial\cdot\epsilon)\eta_{\mu\nu}\label{11}
\end{align}
両辺を$\partial^\nu$を作用させると
\begin{align}
    \partial^\nu\partial_\mu \epsilon_\nu + \partial^2 \epsilon_\mu &= \frac{2}{d}\partial^\nu\eta_{\mu\nu}(\partial\cdot\epsilon)\\
    \partial_\mu(\partial\cdot\epsilon) + \square \epsilon_\mu &= \frac{2}{d}\partial_\mu(\partial\cdot\epsilon)\\
\end{align}
これの$\mu$と$\nu$を入れ替えたものを用意し、辺辺足しあげると
\begin{align}
    2\partial_\mu\partial_\nu(\partial\cdot\epsilon) + 2\square \epsilon_\mu &= \frac{4}{d}\partial_\mu\partial_\nu(\partial\cdot\epsilon)\\
    d \partial_\mu\partial_\nu(\partial\cdot\epsilon) + \square \eta_{\mu\nu}(\partial \cdot \epsilon) &= 2\partial_\mu\partial_\nu(\partial\cdot\epsilon)\\
    \eta_{\mu\nu}\square(\partial\cdot\epsilon) +(d-2)\partial_\mu\partial_\nu(\partial\cdot\epsilon)&=0\label{14}
\end{align}
$\eta^{\mu\nu}$をかけてトレースをとると
\begin{align}   
    \square(\partial\cdot\epsilon) +(d-2)\square(\partial\cdot\epsilon)&=0\\
    (d-1)\square(\partial\cdot\epsilon)&=0
\end{align}
(\ref{11})式に$\partial_\rho $を作用させ、添え字を入れ替えたもの3式を用意し、辺辺足し引きすることにより、次の式を得る。
\begin{align}
    2\partial_\rho\partial_\mu \epsilon_\nu &= \frac{2}{d}\qty(\eta_{\mu\rho}\partial_\nu + \eta_{\nu\rho}\partial_\mu - \eta_{\mu\nu}\partial_\rho)(\partial\cdot\epsilon)\label{18}
\end{align}
\section{2次元の共形場理論}
二次元の共形場理論に限って考える。計量は$g_{00}=1,g_{11}=1,g_{01}=g_{10}=0$のため(\ref{11})より次の関係が成り立つ。
\begin{align}
    \partial_1\epsilon_1 &= \partial_2\epsilon_2\\
    \partial_1\epsilon_2 &= -\partial_2\epsilon_1
\end{align}
この式は複素解析におけるコーシー・リーマンの方程式と同じ形をしている。ここで$z=x^1+ix^2,\overline{z}=x^1-ix^2$と定義する。また、$\epsilon(z,\overline{z})=\epsilon_1(x^1,x^2)+i\epsilon_2(x^1,x^2)$と定義する。このとき、$\epsilon$は$z$のみに依存する正則関数である。
つまり、$z'=z+\epsilon(z)$と変換するとき、これは共形変換となる。すなはち、無限小変換は2次元の複素平面上の正則関数によって与えられる。
実際に、複素平面における任意の正則変換は共形変換であることを示す。
共形変換の条件を満たすか確認する。線素は次のように表される。
\begin{align}
    ds^2 &= dx^1 + dx^2\\
    &= dz d\overline{z}  
\end{align}
ここで、$z \to z' = f(z), \overline{z} \to \overline{z}' = \overline{f}(\overline{z})$と変換する。このとき線素は
\begin{align}
    ds^2 &= dz' d\overline{z}'\\
    &= \pdv{f}{z}\pdv{\overline{f}}{\overline{z}} dz d\overline{z}\\
    &= \qty|\pdv{f}{z}| ds^2
\end{align}
となる。したがって、任意の正則関数は共形変換を与えることがわかる。
\\
無限小変換から場の無限小変換を求める。そのためにまず、無限小変換に対してローラン展開をする。
\begin{align}
    z' &= z + \epsilon(z)=z+\sum_{n\in \mathcal{Z}}\epsilon_n\cdot(-z^{n+1})\\
    \overline{z'}&=\overline{z}+\overline{\epsilon}(\epsilon)=\overline{z}+\sum_{n\in \mathcal{Z}}\overline{\epsilon}_n\cdot(-\overline{z}^{n+1})
\end{align}
無次元スカラー場を考える。
\begin{align}
    \phi'(z',\overline{z'})=\phi(z,\overline{z})
\end{align}
場の無限小変換を求める。
\begin{align}
    \delta \phi(z,\overline{z}) &= \phi'(z,\overline{z}) - \phi(z,\overline{z})\\
    &= -\epsilon(z)\partial_z \phi(z,\overline{z}) - \overline{\epsilon}(\overline{z})\partial_{\overline{z}} \phi(z,\overline{z})\\
    &= -\sum_{n\in \mathcal{Z}}\epsilon_n\cdot(-z^{n+1})\partial_z \phi(z,\overline{z}) - \sum_{n\in \mathcal{Z}}\overline{\epsilon}_n\cdot(-\overline{z}^{n+1})\partial_{\overline{z}} \phi(z,\overline{z})\\
    &= \sum_{n\in \mathcal{Z}}\epsilon_n\cdot l_n \phi(z,\overline{z}) + \sum_{n\in \mathcal{Z}}\overline{\epsilon}_n\cdot \overline{l}_n \phi(z,\overline{z})  
\end{align}
ただし、生成子は$l_n=-z^{n+1}\partial,\overline{l}_n=-\overline{z}^{n+1}\overline{\partial}$という表式をとる。nが整数なので、無限に多くの生成子が存在することになる。
\\
次にこれらの交換関係を求める。
\begin{align}
    [l_n,l_m] 
    &= (m-n)l_{n+m}\\
    [\overline{l}_n,\overline{l}_m] 
    &= (m-n)\overline{l}_{n+m}\\
    [l_n,\overline{l}_m] &= 0
\end{align}
これらの代数関係はwiit algebraと呼ばれる。
\\
\subsection{ネーターの定理}
ネーターの定理
を用いて保存量を求める。まず、変分を求める。
まずこのような座標と場の変換を考える。
\begin{align}
    x'^\mu=x^\mu+\omega_a\frac{\delta x^\mu}{\delta \omega_a}
    \\
    \Phi'(x')=\Phi(x)+\omega_a \frac{\delta \mathcal{F}}{\delta \omega_a}\label{39}
\end{align}
このとき同一の時空点における場の変化は次のように表される。
\begin{align}
    \delta \Phi(x) &= \Phi'(x)-\Phi(x)\\
    &= \Phi'(x') - \Phi(x') + \Phi'(x) - \Phi'(x')\\
    &= \omega_a\qty(\frac{\delta \mathcal{F}}{\delta \omega_a} - \frac{\delta x^\mu}{\delta \omega_a}\partial_\mu \Phi(x))
\end{align}
簡単のためにスカラー場の場合を考えてみる。
スカラー場の場合は、作用の変分は次のように表される。まず$\omega_a$として定数とする。
\begin{align}
    \delta S[\Phi] &=\int d^dx \mathcal{L}(\Phi'(x),\partial_\mu \Phi'(x)) - \int d^dx \mathcal{L}(\Phi(x),\partial_\mu \Phi(x))\\
    &= \int d^dx \qty(\pdv{\mathcal{L}}{\Phi}\delta \Phi + \pdv{\mathcal{L}}{(\partial_\mu \Phi)}\delta(\partial_\mu \Phi))\\
    &= \int d^dx \qty(\pdv{\mathcal{L}}{\Phi}\delta \Phi + \pdv{\mathcal{L}}{(\partial_\mu \Phi)}\partial_\mu(\delta \Phi))\\
    &= \int d^dx \qty(\pdv{\mathcal{L}}{\Phi} - \partial_\mu\qty(\pdv{\mathcal{L}}{(\partial_\mu \Phi)}))\delta \Phi + \int d^dx \partial_\mu\qty(\pdv{\mathcal{L}}{(\partial_\mu \Phi)}\delta \Phi)\\
   &= \int d^dx \qty(\pdv{\mathcal{L}}{\Phi} - \partial_\mu\qty(\pdv{\mathcal{L}}{(\partial_\mu \Phi)}))\qty( - \frac{\delta x^\mu}{\delta \omega_a}\partial_\mu \Phi(x))\omega_a\\
 \end{align}
   表面項は無視できるとする。このとき、この変換に対して対称性があるとき、$\delta S=0$となる。
   つぎに$\omega_a$を定数ではなく、座標依存にした場合を考える。
   このとき、先ほど表面項として無視した部分が重要になる。
   \begin{align}
    \delta S[\Phi] 
    &= \int d^dx \qty(\pdv{\mathcal{L}}{\Phi}\delta \Phi + \pdv{\mathcal{L}}{(\partial_\mu \Phi)}\partial_\mu(\delta \Phi))\\
    &= \int d^dx \qty(\pdv{\mathcal{L}}{\Phi} - \partial_\mu\qty(\pdv{\mathcal{L}}{(\partial_\mu \Phi)}))\delta \Phi + \int d^dx \partial_\mu\qty(\pdv{\mathcal{L}}{(\partial_\mu \Phi)}\delta \Phi)\\
   &= \int d^dx \qty(\pdv{\mathcal{L}}{\Phi} - \partial_\mu\qty(\pdv{\mathcal{L}}{(\partial_\mu \Phi)}))\qty( - \frac{\delta x^\mu}{\delta \omega_a}\partial_\mu \Phi(x))\omega_a\noindent \\
   & \ \ + \int d^dx \partial_\mu\qty(\pdv{\mathcal{L}}{(\partial_\mu \Phi)}\qty(- \frac{\delta x^\mu}{\delta \omega_a}\partial_\mu \Phi(x))\omega_a)\\
   &= \int d^dx \qty(\pdv{\mathcal{L}}{\Phi} - \partial_\mu\qty(\pdv{\mathcal{L}}{(\partial_\mu \Phi)}))\qty( - \frac{\delta x^\mu}{\delta \omega_a}\partial_\mu \Phi(x))\omega_a\notag\\
   & \ \ + \int d^dx \partial_\mu\qty(\pdv{\mathcal{L}}{(\partial_\mu \Phi)}\qty(- \frac{\delta x^\mu}{\delta \omega_a}\partial_\mu \Phi(x)))\omega_a\notag \\
   & \ \ + \int d^dx \pdv{\mathcal{L}}{(\partial_\mu \Phi)}\qty(- \frac{\delta x^\mu}{\delta \omega_a}\partial_\mu \Phi(x))\partial_\mu\omega_a\\
   \end{align}
   変換パラメータ $\omega_a$ を時空点 $x$ に依存する関数とみなした場合、対称性の帰結として、作用の変分は $\partial_\mu \omega_a$ を含む項のみが残るはずである(運動方程式が満たされている場合)。具体的には、大域的変換でのラグランジアンの変化分 $\delta \mathcal{L} = \partial_\mu K^\mu$ (今回の並進では $K^\mu = \mathcal{L} \delta x^\mu$)を考慮すると、局所変換に対する作用の変分は以下のように整理できる。$$\delta S = - \int d^d x \, j_a^\mu \partial_\mu \omega_a$$ここで、$j_a^\mu$ は全微分項の寄与 $K^\mu$ を差し引いて定義されたカレントである
\begin{align}
    \delta S[\Phi] 
    &= \int d^dx \qty(\pdv{\mathcal{L}}{\Phi} - \partial_\mu\qty(\pdv{\mathcal{L}}{(\partial_\mu \Phi)}))\qty( - \frac{\delta x^\mu}{\delta \omega_a}\partial_\mu \Phi(x))\omega_a\notag\\
    & \ \ + \int d^dx \partial_\mu\qty(\pdv{\mathcal{L}}{(\partial_\mu \Phi)}\qty(- \frac{\delta x^\mu}{\delta \omega_a}\partial_\mu \Phi(x)))\omega_a\notag \\
    & \ \ + \int d^dx \pdv{\mathcal{L}}{(\partial_\mu \Phi)}\qty(- \frac{\delta x^\mu}{\delta \omega_a}\partial_\mu \Phi(x))\partial_\mu\omega_a\notag\\
  &\ \  + \int d^dx\  -\omega_a\frac{\delta x^\mu}{\delta \omega_a }\partial_\mu \mathcal{L}\label{55}
\end{align}
最後の項に対して部分積分を行い$\partial_\mu \omega_a$に比例する項を出現させる。表面項が現れるがこれは無視できるとする。
ここで任意の関数$\omega_a(x)$に対して0にならなければならないので、$\partial_\mu \omega_a$の係数部分が保存していなければならない。
%https://gemini.google.com/share/30b5293073dd
ここで、$\partial_\mu \omega_a$の係数を$-j_a^\mu$とするとカレント$j_a^\mu$の表式は次のようになっている。
\begin{align}
    j_a^\mu &= \qty(\pdv{\mathcal{L}}{(\partial_\mu \Phi)}\partial_\nu \Phi - \delta^\mu_\nu \mathcal{L})\frac{\delta x^\nu}{\delta \omega_a}\label{56}
\end{align}

    %次に作用の変分を求める。
%\begin{align}
%    \delta S &= S[\Phi'(x)] - S[\Phi(x)]\\
 %   &= \int d^dx \mathcal{L}(\Phi'(x),\partial_\mu \Phi'(x)) - \int d^dx \mathcal{L}(\Phi(x),\partial_\mu \Phi(x))\\
  %  &= \int d^dx \qty(\pdv{\mathcal{L}}{\Phi}\delta \Phi + \pdv{\mathcal{L}}{(\partial_\mu \Phi)}\delta(\partial_\mu \Phi))\\
   % &= \int d^dx \qty(\pdv{\mathcal{L}}{\Phi}\qty(\frac{\delta \mathcal{F}}{\delta \omega_a} - \frac{\delta x^\mu}{\delta \omega_a}\partial_\mu \Phi(x))-\frac{\delta x^\nu}{\delta \omega_a}\partial_\nu\qty(\pdv{\mathcal{L}}{(\partial_\mu \Phi)})\partial_\mu \Phi(x))\omega_a\\
   % & \ \ \ \ +\pdv{\mathcal{L}}{(\partial_\mu \Phi)}\qty(\frac{\delta \mathcal{F}}{\delta \omega_a} - \frac{\delta x^\mu}{\delta \omega_a}\partial_\mu \Phi(x))\partial_\mu\omega_a\\
    %&=\int d^dx \pdv{\mathcal{L}}{(\partial_\mu \Phi)}\frac{\delta \mathcal{F}}{\delta \omega_a} \partial_\mu \omega_a+\qty(- \pdv{\mathcal{L}}{(\partial_\mu \Phi)}\frac{\delta x^\nu}{\delta \omega_a}\partial_\nu \Phi(x)+\pdv{\mathcal{L}}{\Phi}\frac{\delta x^\nu}{\delta \omega_a}\Phi(x)\delta^\nu_\mu+\frac{\delta x^\nu}{\delta \omega_a}\pdv{\mathcal{L}}{(\partial_\mu\Phi)}\partial_\nu\Phi(x))\partial_\mu\omega_a\\
%\end{align}
(\ref{55})式より保存の式の式が成り立つことを確認する。微分がついていない項は変分原理より0になるので、微分項のみを考える。
\begin{align}
    0 &=-\int d^dx j_a^\mu \partial_\mu \omega_a\\
    &=-\int d^dx \partial_\mu (j_a^\mu \omega_a) + \int d^dx \omega_a \partial_\mu j_a^\mu  
\end{align}
この変換に対して系が対称性を持つとは、作用の変分 $\delta S$ がゼロ、もしくは全微分項(表面項)の積分にとどまることを意味する。表面項が境界条件により無視できる場合、大域的な変換($\omega_a$ が定数)に対して作用は不変($\delta S = 0$)となる。
\begin{align}
    \partial_\mu j^\mu_a=0\label{59}
\end{align}
ここから電荷を定義する。
\begin{align}
    Q_a=\int d^{d-1}x j^0_a
\end{align}
これは保存の式(\ref{59})を用いることにより、保存していることが分かる。
\begin{align}
    \frac{d}{d t}Q_a=\int d^{d-1}\partial_0j_a^0=-\int d^{d-1}x \partial_i j^i=0 
\end{align}
最後はガウスの発散定理を用いて、無限遠においてカレントが0になることを使った。
このカレントに反対称なテンソルを加えても保存の式は成り立つことを確認する。
\begin{align}
    j_a^\mu \rightarrow j_a^\mu +\partial_\nu B_a^{\nu\mu}\ ,\ B_a^{\nu\mu}=-B_a^{\mu\nu} 
\end{align}
この変換に対して、$\partial_\mu$を作用させると2項目に対して、対称テンソルと反対称テンソルにより0になるので、保存の式は成り立っている。
\\
次にエネルギーモーメントテンソルを考える。
$x'^\mu=x^\mu +\omega^\mu(x)$という任意の座標変換を考える。
\begin{align}
    \delta S=\int d^dxT^{\mu\nu}\partial_\mu \omega_\nu
\end{align}
特に並進の$\omega^\mu=const$の場合ではエネルギー運動量テンソルの表式が次のようになる。
\begin{align}
    T_c^{\mu\nu}=-\eta^{\mu\nu}\mathcal{L}+\pdv{\mathcal{L}}{\partial_\mu \Phi}\partial^\nu \Phi
\end{align}
この様に定義したエネルギー運動量テンソルは添字$(\mu\nu)$に関して対称ではない。しかし、カレントの場合の手順
と同様にして対称化することができる。
\begin{align}
    T_B^{\mu\nu}=T^{\mu\nu}+\partial_\rho B^{\rho \mu\nu} \ \ B^{\rho\mu\nu}=-B^{\mu \rho \nu}
\end{align}
共形変換$x'^\mu =x^\mu +\epsilon^\mu (x)$
このとき、カレントとエネルギー運動量テンソルの関係は次のようになっている。
\begin{align}
    j_\mu=T_{\mu\nu}\epsilon^\nu\label{66}
\end{align}
並進は共形変換であるので、カレントの保存は次を示す。今は並進のことを考えてるので、$\epsilon^\nu$は定数である。
\begin{align}
    0=\partial^\mu j_\mu=\partial^\mu(T_{\mu\nu}\epsilon^\nu)=(\partial^\mu T_{\mu \nu})\epsilon^\nu
\end{align}
$\epsilon$は任意であるので、$\partial^\mu T_{\mu\nu}=0$が得られる。一般の共形変換に対して、つまり $x'^\mu=x^\mu +\epsilon^\mu(x)$の場合を考える。保存の式から次のことを示せる。
\begin{align}
    0=\partial^\mu j_\mu =(\partial^\mu T_{\mu\nu})\epsilon^\nu + T_{\mu\nu}\partial^\mu \epsilon^\nu
    &= T_{\mu\nu}\partial^\mu \epsilon^\nu\\
    &= T_{\mu\nu}\frac{1}{2}(\partial^\mu \epsilon^\nu + \partial^\nu \epsilon^\mu)\label{69}\\
    &= T_{\mu\nu}\frac{1}{d}(\partial \cdot \epsilon)\eta^{\mu\nu}\label{70}\\
    &= \frac{1}{d}(\partial \cdot \epsilon)T^{\ \mu}_\mu\label{71}
\end{align}
(\ref{69})式ではエネルギー運動量テンソルが対称であることを用いた。(\ref{70})式では共形キリング方程式(\ref{11})式を用いた。
(\ref{71})式より、エネルギー運動量テンソルのトレースが0であるか$\epsilon$がキリングベクトル($\partial \cdot \epsilon=0$)である場合に保存の式が成り立つことがわかる。共形キリングベクトルの場合は$\epsilon$はキリングベクトルではないので、エネルギー運動量テンソルのトレースが0である必要がある。
\subsection{ward-Takahashiの関係式}
場を量子化した場合を考える。今までは古典的に連続対称性にフォーカスをあてていたが、量子レベルでは相関関数が重要になる。
\subsubsection{相関関数}
まず自由場のラグランジアン密度はローレンツ不変性を満たすことに注意して次のように書くことが出来る。
\begin{align}
    \mathcal{L}=\frac{1}{2}\partial_\mu \phi \partial^\mu \phi - \frac{1}{2}m^2 \phi^2
\end{align}
経路積分は次のように書き下すことが出来る。
\begin{align}
    Z_0(J)\equiv \braket{0}{0}_J=\int \mathcal{D}\phi \exp\qty[i\int d^4x \qty(\mathcal{L}+J(x)\phi(x))]
\end{align}    
ここで$J$は外部から加えられるソース項である。相関関数は次のようにして求めることが出来る。
$[D\phi]$は可能なすべての場の配置にわたる積分を表している。
\begin{align}
    \mel**{0}{T\phi(x_1)\phi(x_2)\cdots \phi(x_n)}{0}=\frac{1}{i^n}\frac{1}{Z_0(0)}\qty(\frac{\delta}{\delta J(x_1)}\frac{\delta}{\delta J(x_2)}\cdots \frac{\delta}{\delta J(x_n)})Z_0(J)\Bigg|_{J=0}
\end{align}
左辺のTは時間順序を表している。相関関数は場の量子論において重要な役割を果たす。
又相関関数は次のようにも書くことが出来る。
\begin{align}
    \mel**{0}{T\phi(x_1)\phi(x_2)\cdots \phi(x_n)}{0}=\frac{\int \mathcal{D}\phi \ \phi(x_1)\phi(x_2)\cdots \phi(x_n) \exp\qty[i\int d^4x \ \mathcal{L}]}{\int \mathcal{D}\phi \exp\qty[i\int d^4x \ \mathcal{L}]}
\end{align}
経路積分の中に含まれる場は自由に入れ替えることが出来る。したがって時間順序は自動的に満たされている。
\begin{align}
      \mel**{0}{T\phi(x_1)\phi(x_2)\cdots \phi(x_n)}{0}=\ev{\phi(x_1)\phi(x_2)\cdots \phi(x_n)}
\end{align}
ここで$\ev{\cdots}$は真空期待値を表している。\\
連続対称性は相関関数にどのような影響を与えるかを考える。
連続変換により$x\rightarrow x'$に$\phi\rightarrow \phi'$に変化したとする。
このとき次のような関係が成り立つ。
\begin{align}
    \ev{\phi(x'_1)\phi(x'_2)\cdots \phi(x'_n)}=&\frac{1}{Z_0}\int \mathcal{D}\phi \ \phi(x_1)\phi(x_2)\cdots \phi(x_n) \exp\qty[i\int d^4x \ \mathcal{L}]\\
    =&\frac{1}{Z_0}\int \mathcal{D}\phi' \ \phi'(x_1)\phi(x_2)\cdots \phi(x_n) \exp\qty[i\int d^4x \ \mathcal{L}]\\
=&\frac{1}{Z_0}\int \mathcal{D}\phi \ \mathcal{F}(\phi(x_1))\mathcal{F}(\phi(x_2))\cdots \mathcal{F}(\phi(x_n)) \exp\qty[i\int d^4x' \ \mathcal{L}']\\
=&\ev{\mathcal{F}(\phi(x_1))\mathcal{F}(\phi(x_2))\cdots \mathcal{F}(\phi(x_n))}
\end{align}  
2つ目の等号では積分変数を変えているに過ぎない。3つ目の等号では場の変換(\ref{39})を用いている。
\subsubsection{ward-Takahashiの関係式}
ネーターの定理の量子論版がward-Takahashiの関係式である。まず無限小変換を考える。
\begin{align}
    \Phi'(x)=\Phi(x)-i\omega_a G_a\Phi(x)
\end{align}
ここで$G_a$は変換の生成子である。作用の変分は次のようになる。
相関関数は対称性に対して不変であるので、次のように書くことが出来る。
\begin{align}
    \ev{X}=\frac{1}{Z_0}\int \mathcal{D}\Phi \ (X+\delta X) \exp\qty[iS(\Phi)-i\int d^dx \omega_a(x)\partial_\mu j_a^\mu (x)]%この部分怪しい
\end{align}
ここで$X=\Phi(x_1)\Phi(x_2)\cdots \Phi(x_n)$である。$S[\Phi]=\int d^dx \mathcal{L}[\Phi]$を表している。最後の部分はネーターの定理を用いて作用の変分を表している。
これを展開すると次のようになる。
\begin{align}
   \ev{X}=\ev{X}+\ev{\delta X}-i\int d^dx \ \omega_a(x)\ev{X\partial_\mu j_a^\mu (x)}\\
\ev{\delta X}=i\int d^dx \ \omega_a(x)\ev{X\partial_\mu j_a^\mu (x)}
\end{align}
変分$\delta X$は次のように書くことが出来る。
\begin{align}
    \delta X=&-i\sum_{j=1}^n (\Phi(x_1)\cdots G_a\Phi(x_j)\cdots \Phi(x_n))\omega_a(x_i)\\
    =&-i\int d^dx \ \omega_a(x)\sum_{j=1}^n \delta(x-x_j)\Phi(x_1)\cdots G_a \Phi(x_j)\cdots \Phi(x_n)
\end{align}
これら2つの表現が任意の関数$\omega_a(x)$に対して成り立つためには次の関係が成り立つ必要がある。
\begin{align}
    \ev{X\partial_\mu j_a^\mu (x)}=-\sum_{j=1}^n \delta(x-x_j)\ev{\Phi(x_1)\cdots G_a \Phi(x_j)\cdots \Phi(x_n)}\label{87}
\end{align}
この関係式はward-Takahashiの関係式と呼ばれる。左辺を見ると、場の引数と等しくなったときに発散することが分かる。逆に、場が挿入されていない点においてはカレントが保存していることが分かる。
\\
この関係式により量子状態のヒルベルト空間に対する対称性のある変換の生成子を定義することが出来る。
それを保存電荷と呼び$Q_a$とかく。空間成分でカレントを積分することにより定義される。
\begin{align}
    Q_a=\int d^{d-1}x \ j_a^0(x)
\end{align}
ストークスの定理を用いることにより、この保存電荷は次のように書き換えることが出来る。
\begin{align}
    \int d^{d}x \partial_\mu j_a^\mu (x)=\int dS_\mu j^\mu_a = Q(t_1)-Q(t_0)
\end{align}
ただし$t_1>t_0$。この保存電荷は場に対して次のように作用する。
この保存電荷は場に対して次のように作用する。
\begin{align}
    \ev{Q_a(t_1)\Phi(x_1)\cdots \Phi(x_n)} - \ev{\Phi(x_1)\cdots \Phi(x_n) Q_a(t_0)}=&-\ev{G_a\Phi(x_1)\cdots \Phi(x_n)}\\
    [Q_a,\Phi(x)]=&-i G_a \Phi(x)
\end{align}
ここで、$t_1> x^0 > t_0$であることに注意する。
この関係式は保存電荷が場に対してどのように作用する
かを示している。したがって、保存電荷は変換の生成子であることが分かる。
\subsection{プライマリ場}
共形場理論において特別な役割を果たす場をプライマリ場というものを定義する。
\\
まず準プライマリ場を定義する。大域的共形変換$z\rightarrow \omega(z),\overline{z}\rightarrow \overline{\omega}(\overline{z})$の下で、場$\phi(x)$が次のように変換するとき、その場を準プライマリ場と呼ぶ。
\begin{align}
    \phi'(\omega,\overline{\omega})=\qty(\dv{\omega}{z})^{-h}\qty(\dv{\overline{\omega}}{\overline{z}})^{-\overline{h}}\phi(z,\overline{z})
\end{align}
ここで、$h,\overline{h}$は場の共形次元と呼ばれる無次元数である。
\\
次に無限小変換の場合を考える。
\begin{align}
    z'&=z+\epsilon(z)\\
    \overline{z'}&=\overline{z}+\overline{\epsilon}(\overline{z})
\end{align}
このとき、準プライマリ場の無限小変換は次のようになる。
\begin{align}
    \delta_{\epsilon,\overline{\epsilon}} \phi(z,\overline{z}) &= \phi'(z,\overline{z}) - \phi(z,\overline{z})\\
    &= \phi'(z'-\epsilon(z),\overline{z}'-\overline{\epsilon}(\overline{z}) ) - \phi(z,\overline{z})\\
    &= -h\pdv{\epsilon(z)}{z}\phi(z,\overline{z}) - \overline{h}\pdv{\overline{\epsilon}(\overline{z})}{\overline{z}}\phi(z,\overline{z}) - \epsilon(z)\partial_z \phi(z,\overline{z}) - \overline{\epsilon}(\overline{z})\partial_{\overline{z}} \phi(z,\overline{z})\\
    &=-(h\phi \partial_z \epsilon + \epsilon \partial_z \phi) - (\overline{h}\phi \partial_{\overline{z}} \overline{\epsilon} + \overline{\epsilon} \partial_{\overline{z}} \phi)
\end{align}
プライマリ場は準プライマリ場のうち、任意の共形変換に対して上記の変換則を満たす場である。
この定義よりプライマリ場は準プライマリ場の条件を満たす。しかし、その逆は成り立たない。
\\
\subsubsection{2点関数}この定義により、準プライマリ場の相関関数に制約が課されることを確認する。
並進対称性より、2点関数は座標の差の関数である。
\begin{align}
    \ev{\phi_1(z)\phi_2(w)}=g(z-w)
\end{align}
ここで、大域的な共形変換$z\rightarrow \lambda z$, $w\rightarrow \lambda w$を考える。
このとき、2点関数は次のように変換する。
\begin{align}
    \ev{\phi_1(\lambda z)\phi_2(\lambda w)}=&\lambda^{-h_1 - h_2}\lambda^{-\overline{h}_1 - \overline{h}_2}\ev{\phi_1(z)\phi_2(w)}\\
    =&\lambda^{-h_1 - h_2}\lambda^{-\overline{h}_1 - \overline{h}_2}g(z-w)\label{101}
\end{align}
一方で、2点関数は次のようにも変換する。
\begin{align}
    \ev{\phi_1(\lambda z)\phi_2(\lambda w)}=&g(\lambda z - \lambda w)\\
    =&g(\lambda (z-w))\label{103}
\end{align}
(\ref{101})式と(\ref{103})式を見比べ、これが任意の定数$\lambda$に対して成り立つためには、$g(z-w)$は次の形をとる必要がある。
\begin{align}
    g(z-w)=\frac{C_{12}}{(z-w)^{h_1 + h_2}(\overline{z}-\overline{w})^{\overline{h}_1 + \overline{h}_2}}
\end{align}
ここで、$C_{12}$は定数である。
更に,
反転変換$z\rightarrow -\frac{1}{z},\omega\rightarrow -\frac{1}{\omega}$を考える。このとき、2点関数は次のように変換する。
\begin{align}
    \ev{\phi_1\qty(-\frac{1}{z})\phi_2\qty(-\frac{1}{w})}=&\qty(\frac{1}{z^2})^{-h_1}\qty(\frac{1}{w^2})^{-h_2}\qty(\frac{1}{\overline{z}^2})^{-\overline{h}_1}\qty(\frac{1}{\overline{w}^2})^{-\overline{h}_2}\ev{\phi_1(z)\phi_2(w)}\\
    =&\frac{(z)^{2h_1}(w)^{2h_2}(\overline{z})^{2\overline{h}_1}(\overline{w})^{2\overline{h}_2}C_{12}}{(z-w)^{h_1 + h_2}(\overline{z}-\overline{w})^{\overline{h}_1 + \overline{h}_2}}
\end{align}\label{106}
一方で、2点関数は次のようにも変換する。
\begin{align}
    \ev{\phi_1\qty(-\frac{1}{z})\phi_2\qty(-\frac{1}{w})}=&g\qty(-\frac{1}{z} + \frac{1}{w})\\
    =&g\qty(\frac{z-w}{zw})\\
    =&\frac{C_{12}}{(\frac{z-w}{zw})^{h_1 + h_2}(\frac{\overline{z}-\overline{w}}{\overline{z}\overline{w}})^{\overline{h}_1 + \overline{h}_2}}\\
    =&\frac{(z)^{h_1 + h_2}(w)^{h_1 + h_2}(\overline{z})^{\overline{h}_1 + \overline{h}_2}(\overline{w})^{\overline{h}_1 + \overline{h}_2}C_{12}}{(z-w)^{h_1 + h_2}(\overline{z}-\overline{w})^{\overline{h}_1 + \overline{h}_2}}\label{110}
\end{align}
(\ref{106})式と(\ref{110})式を見比べ、これが任意の$z,w$に対して成り立つためには、$h_1=h_2$かつ$\overline{h}_1=\overline{h}_2$が必要である。
\subsubsection{3点関数}
次に準プライマリ場の3点関数を考える。先ほどと同様に並進対称性より、次のように書くことが出来る。
\begin{align}
    \ev{\phi_1(x_1)\phi_2(x_2)\phi_3(x_3)}&=g(x_1 - x_2,x_2 - x_3,x_3 - x_1)\\
    &=g(x_{12},x_{23},x_{31})
    \end{align}
ここで、$x_{ij}=x_i - x_j$と定義した。スケール変換$\lambda$の下で、3点関数は次のように変換する。
\begin{align}
    g(x_{12},x_{23},x_{31})&=\lambda^{-(h_1 + h_2 + h_3)}g(\lambda x_{12},\lambda x_{23},\lambda x_{31})\label{116}
\end{align}
$\lambda r_{12}=r_{12}^0$とすると、(\ref{116})式は次のように書ける。
\begin{align}
    g(x_{12},x_{23},x_{31})&=\frac{1}{(r_{12})^{h_1 + h_2 + h_3}}g\qty(r_{12}^0,\frac{r_{23}r_{12}^0}{r_{12}},\frac{r_{31}r_{12}^0}{r_{12}})\label{118}
\end{align}
右辺は$\frac{r_{23}}{r_{12}}$,$\frac{r_{31}}{r_{12}}$が十分小さい場合展開することが出来る。
\begin{align}
    g(x_{12},x_{23},x_{31})&=\sum_{a,b,c}\frac{C_{abc}}{r_{12}^ar_{23}^br_{31}^c}
\end{align}
次に特殊共形変形$x'^{\mu} =\frac{x^\mu-b^\mu x^2}{1-2bx+b^{2} x^{2}}$を考える。
そのもとでの3点関数の変換の変換性は次のようになる。特殊共形変換は反転→平行移動→反転で、表現されることを用いる。\footnote{
   簡単のため1次元の場合を考える。まず、平行移動$x' = x - b$の変換は$\pdv{x'}{x}=1$である。次に
    反転による変換は$\pdv{y}{x}=-\frac{1}{x^2}$
    である。したがって、特殊共形変換の
    \begin{align}
    \pdv{x'}{x}=&\pdv{x'}{y}\pdv{y}{x}\\
    =&-\frac{1}{(1/x-b)^2}\cdot \qty(-\frac{1}{x^2})\\
    =&\frac{1}{(1-2bx+b^2 x^2)^2}
    \end{align}
    }
\begin{align}
  \frac{C_{abc}}{r_{12}^a r_{23}^b r_{31}^c}=&\frac{C_{abc}}{r_{12}^{'a} r_{23}^{'b} r_{31}^{'c}\cdot (1-2b\cdot x_1 + b^2 x_1^2)^{h_1}(1-2b\cdot x_2 + b^2 x_2^2)^{h_2}(1-2b\cdot x_3 + b^2 x_3^2)^{h_3}}\label{a}\\
  =&\frac{C_{abc}}{r_{12}^{a} r_{23}^{b} r_{31}^{c}}\cdot\frac{\gamma_1^{\frac{a+c}{2}}\gamma_2^{\frac{a+b}{2}}\gamma_3^{\frac{b+c}{2}}}{\gamma_1^{h_1}\gamma_2^{h_2}\gamma_3^{h_3}}\label{b}
\end{align}
ここで、$\gamma_i=1-2b\cdot x_i + b^2 x_i^2$とした。\footnote{(\ref{a})式から
(\ref{b})式への変換は特殊共形変換が反転と平行移動を用いて表現されることを用いる。まず、反転変換$x^\mu \rightarrow \frac{x^\mu}{x^2}$の下での距離の変換を考える。
\begin{align}
    r_{ij}^{'2}=\qty(\frac{x_i^\mu}{x_i^2}-\frac{x_j^\mu}{x_j^2})^2=\frac{r_{ij}^2}{x_i^2 x_j^2}
\end{align}
次に平行移動$x^\mu \rightarrow x^\mu - b^\mu$の下での距離の変換を考える。
\begin{align}
    r_{ij}^{''2}=\qty(\frac{x_i^\mu}{x_i^2}-b^\mu - \frac{x_j^\mu}{x_j^2}+b^\mu)^2=r_{ij}^{'2}
\end{align}
最後に再び反転変換を行うと次のようになる。
\begin{align}
    r_{ij}^{'''2}=\qty(\frac{\frac{x_i^\mu}{x_i^2}-b^\mu}{\qty(\frac{x_i^\mu}{x_i^2}-b^\mu)^2}-\frac{\frac{x_j^\mu}{x_j^2}-b^\mu}{\qty(\frac{x_j^\mu}{x_j^2}-b^\mu)^2})^2=\frac{r_{ij}^{'2}}{\qty(\frac{x_i^\mu}{x_i^2}-b^\mu)^2\qty(\frac{x_j^\mu}{x_j^2}-b^\mu)^2}
\end{align}
これらを組み合わせると次のようになる。
\begin{align}
    r_{ij}^{'''2}=\frac{r_{ij}^2}{x_i^2 x_j^2 \qty(\frac{x_i^\mu}{x_i^2}-b^\mu)^2\qty(\frac{x_j^\mu}{x_j^2}-b^\mu)^2}=\frac{r_{ij}^2}{(1-2b\cdot x_i + b^2 x_i^2)(1-2b\cdot x_j + b^2 x_j^2)}
\end{align}
}
この関係式が任意の$b$に対して成り立つためには、次の関係式が成り立つ必要がある。
%https://gemini.google.com/share/b225813d907f
\begin{align}
    h_1=&\frac{a+c}{2}\\
    h_2=&\frac{a+b}{2}\\
    h_3=&\frac{b+c}{2}
\end{align}
これらの式を解くと次のようになる。
\begin{align}
    a=&h_1 + h_2 - h_3\\
    b=&-h_1 + h_2 + h_3\\
    c=&h_1 - h_2 + h_3
\end{align}
したがって、3点関数は次のように書ける。
\begin{align}
    \ev{\phi_1(x_1)\phi_2(x_2)\phi_3(x_3)}=&\frac{C_{123}}{r_{12}^{h_1 + h_2 - h_3}r_{23}^{-h_1 + h_2 + h_3}r_{31}^{h_1 - h_2 + h_3}}
\end{align}

\subsection{共形Ward-Takahashiの関係式}
ward-Takahashiの関係式を共形変換の場合に適用する。
共形不変性の場合のカレントはエネルギー運動量テンソルを用いて
(\ref{66})式は次のように表されていた。
このとき、共形Ward-Takahashiの関係式は次のようになる。
\begin{align}
   \delta_{\epsilon}\ev{X}=\int_D d^dx \ \partial_\mu \ev{T_{\mu\nu}\epsilon^\nu (x)X}
\end{align}
$X$には場の積が入る。$D$は場の挿入点$x_i$を含む任意の領域である。
ここで、2次元の共形場理論では、複素平面と相性が良いので、これを用いて議論を進める。
エネルギー運動量テンソルを複素座標の成分で表す。$dz\wedge d\overline{z}=(dx^1 +idx^2)\wedge(dx^1-idx^2) =-2idx^1\wedge dx^2
$に注意する。
\begin{align}
    \int_D d^2x \ \partial^\mu \ev{T_{\mu\nu}\epsilon^\nu (x)X}=\int_D d^2z\cdot \frac{i}{2} \ \qty(\partial^z \ev{T_{zz}\epsilon(z)X}+\partial^{\overline{z}} \ev{T_{\overline{z}z}\overline{\epsilon}(\overline{z})X}+\partial^z \ev{T_{z\overline{z}}\overline{\epsilon}(\overline{z})X}+\partial^{\overline{z}} \ev{T_{\overline{z}z}\epsilon(z)X}
    )
    \end{align}
ここで$\frac{\partial}{\partial z}=\frac{1}{2}\qty(\pdv{x^1}+\frac{1}{i}\pdv{x^2})$,$\frac{\partial}{\partial \overline{z}}=\frac{1}{2}\qty(\pdv{x^1}-\frac{1}{i}\pdv{x^2}),\epsilon(z)=\epsilon_1+i\epsilon_2,\overline{\epsilon}(\overline{z})=\epsilon_1-i\epsilon_2$と書けることと複素座標での計量に注意すると次のようにエネルギー運動量テンソルを複素座標で表すことが出来る。
\begin{align}
    T_{zz}=&\pdv{x^\mu}{z}\pdv{x^\nu}{z}T_{\mu\nu}=\frac{1}{4}T_{11}-\frac{1}{4}T_{22}-\frac{i}{4}T_{12}-\frac{i}{4}T_{21}=\frac{1}{4}(T_{11}-2iT_{12}-T_{22})\\
\end{align}
以下同様にして次のように計算することが出来る。
\begin{align}
    T_{zz}&=\frac{1}{4}\qty(T_{11}-T_{22}+iT_{12}+iT_{21})=\frac{1}{4}(T_{11}-2iT_{12}-T_{22})\\
    T_{\overline{z}\overline{z}}&=\frac{1}{4}\qty(T_{11}-T_{22}-iT_{12}-iT_{21})=\frac{1}{4}(T_{11}+2iT_{12}-T_{22})\\
    T_{z\overline{z}}&=\frac{1}{4}\qty(T_{11}+T_{22}-iT_{12}+iT_{21})=\frac{1}{4}(T_{11}+T_{22})\\
    T_{\overline{z}z}&=\frac{1}{4}\qty(T_{11}+T_{22}+iT_{12}-iT_{21})=\frac{1}{4}(T_{11}+T_{22})
\end{align}
トレースレス性$T^\mu_\mu=0$より、$T_{z\overline{z}}=T_{\overline{z}z}=0$であることに注意する。
したがって、次のように書ける。
\begin{align}
    \int_D d^2x \ \partial^\mu \ev{T_{\mu\nu}\epsilon^\nu (x)X}=&\int_D d^2z\cdot \frac{i}{2} \ \qty(\partial^z \ev{T_{zz}\epsilon(z)X}+\partial^{\overline{z}} \ev{T_{\overline{z}\overline{z}}\overline{\epsilon}(\overline{z})X})
\end{align}
ここで、共形変換は$z\rightarrow z+\epsilon(z),\overline{z}\rightarrow \overline{z}+\overline{\epsilon}(\overline{z})$であることに注意する。
ガウスの定理を用いると、次のように書くことが出来る。$d^2z=dz\wedge d\overline{z}$の意味であったことに注意して、
\begin{align}
   \delta_{\epsilon,\overline{\epsilon}}\ev{X}=\frac{i}{2}\oint_{\partial D} \qty(d\overline{z} \ev{T_{zz}\epsilon(z)X}-dz \ev{T_{\overline{z}\overline{z}}\overline{\epsilon}(\overline{z})X})
\end{align}
保存の式$\partial^\mu T_{\mu\nu}=0$より、$\partial^{z} \ev{T_{zz}\epsilon(z)X}=0$かつ$\partial^{\overline{z}} \ev{T_{\overline{z}\overline{z}}\overline{\epsilon}(\overline{z})X}=0$である。ここで、複素の場合の計量は非対角成分のみが残っていることに注意すると、$T_{zz}$は正則な関数であり、$T_{\overline{z}\overline{z}}$は反正則な関数であることが分かる。
よって、$T=-2\pi T_{zz},\overline{T}=-2\pi T_{\overline{z}\overline{z}}$と定義すると、共形Ward-Takahashiの関係式は次のように書ける。
\begin{align}
    \delta_{\epsilon,\overline{\epsilon}}\ev{X}=-\frac{1}{2\pi i}\oint_{\partial D} dz \ \epsilon(z) \ev{T(z)X} + \frac{1}{2\pi i}\oint_{\partial D} d\overline{z} \ \overline{\epsilon}(\overline{z}) \ev{\overline{T}(\overline{z})X}
\end{align}
これをプライマリ場の場合に適用することを考えると、任意の局所的な共形変換に対しる相関関数の変化の定義式と捉えることが出来る。それが共形Ward-Takahashiの関係式である。
$X$には場の積が入る。\\簡単のためにここでは場が1つ挿入された場合について考えてみる。
場が1つの場合は、積分領域$D$をいくらでも小さく取れるので、エネルギー運動量テンソルの位置$z$がプライマリ場の位置$z_1$に近づくときの特異的な振る舞いが積分に効いてくる。
そのために次の関係式を持ち出す。以下はコーシーの積分定理を用いている。
\begin{align}
    (\partial_{\omega_i}\epsilon(\omega_i))\phi_i(\omega_i,\overline{\omega}_i)=&\frac{1}{2\pi i}\oint_{C_i} dz \ \frac{\epsilon(z)}{(z-\omega_i)^2}\phi_i(\omega_i,\overline{\omega_i})\\
\epsilon(\omega_i)(\partial_{\omega_i}\phi_i(\omega_i,\overline{\omega}_i))=&\frac{1}{2\pi i}\oint_{C_i} dz \ \frac{\epsilon(z)}{(z-\omega_i)}\partial_{\omega_i}\phi_i(\omega_i,\overline{\omega}_i)
\end{align}
よって共形Ward-Takahashiの関係式の中身は次のように書ける。
\begin{align}
\ev{T_z X}=&\sum_{i=1}^n \qty(\frac{h_i}{(z-z_i)^2}+\frac{1}{z-z_i}\partial_{z_i})\ev{X}+reg.\\
\end{align}
ここで、$reg.$は正則な部分を表している。同様にして反正則な部分についても次のように書ける。
この表現はすなわち、エネルギー運動量テンソルの引数
$z$がプライマリ場の引き位置$z_i$に近づくときの特異的な振る舞いを示している。
これは、ブラケットを取り除くことによりエネルギー運動量テンソルの演算子積展開
を与えている。
\begin{align}
    T(z)\phi_i(w,\overline{w})\sim \frac{h_i}{(z-w)^2}\phi_i(w,\overline{w})+\frac{1}{z-w}\partial_w \phi_i(w,\overline{w})
\end{align}
$\sim$というのは、等式ではなく、特異的な振る舞いを示している。
すなわち、正則な部分は無視している。
しかし、それらの寄与はWard-Takahashiの関係式からは得ることができない。
\subsection{ビラソロ代数}
\subsubsection{動径量子化}
通常の場の量子論では時間方向と空間方向は異なる扱いを受ける。一方、2次元の共形場理論ではユークリッド計量での複素座標を
用いて議論を進めることが多い。このとき、時間方向と空間方向は対等に扱われる。
そのため、時間方向に関する順序付けが出来なくなる。
その代わりに、動径量子化という方法を用いる。動径量子化では、時間方向を複素平面上の原点からの距離$r$で定義する。
まず、閉弦の場合を考える。閉弦の場合、空間方向は周期的である。一方、時間方向は高さのほうこうであり、無限に伸びている。
これらを$\xi=t+ix$という変形により、複素平面上に写す。
ここで、$t$は時間方向、$x$は空間方向である。円柱の周囲を$l$とする。
これをさらに円柱から複素平面への写像$z=e^{2\pi\xi/l}$により、複素平面上に写す。




\end{document}