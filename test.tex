\documentclass[11pt, draft]{article}
\usepackage[utf8]{inputenc}
\usepackage[margin=35truemm]{geometry} % 余白を35mmに設定

% --- 数学関連のパッケージ ---
\usepackage{amsmath}      % 様々な数式環境
\usepackage{amsthm}       % 定理環境
\usepackage{amsfonts}     % 数学用フォント
\usepackage{physics}      % 物理学の数式用マクロ (bra, ket, dvなど)
\usepackage{bm}           % 太字の数式 \bm{}

% --- 図や色、レイアウト関連のパッケージ ---
\usepackage[dvipdfmx]{graphicx, color} % 図の挿入や色の利用
\usepackage{tikz}         % 図形描画
\usetikzlibrary{intersections,calc,arrows.meta}
\usepackage{float}        % 図表の位置を調整 [H]など
\usepackage{siunitx}      % 単位をきれいに表示 \SI{100}{\kilo\gram}など
\usepackage{ascmac}       % itemboxなどの囲み枠

% --- その他 ---
\parindent = 0pt % 全体の段落開始時のインデントをなくす

% --- ハイパーリンク関連 (できるだけ最後に読み込む) ---
\usepackage[
    dvipdfmx,
    bookmarks=true,
    bookmarksnumbered=true,
    bookmarkstype=toc
]{hyperref}
\usepackage{pxjahyper} % hyperrefの日本語文字化け対策
\begin{document}

\title{共形場理論}
\author{小武 龍斗}
\date{\today}
\maketitle
\section{}
\section{共形変換}
まず共形変換の定義を与える。共形変換は、二つの多様体$M,N$の開集合$U,V$に対して、微分可能写像を$\varphi: U \to V$とする。
$g$と$g'$はそれぞれ$M$と$N$の計量テンソルである。
このとき引き戻しが$$\varphi^* g' = \Omega g$$を満たすとき、共形変換であると言う。ここで$\Omega$は$U$上のスカラー関数である。\\
このとき$\phi=x'$とすると$T_pM$の基底ベクトル$\pdv{x^\mu},\pdv{x^\nu}$を代入して$$g'\qty(\phi_*\qty(\pdv{x^\mu}),\phi_*\qty(\pdv{x^\nu})) = \Lambda g\qty(\pdv{x^\mu},\pdv{x^\nu})$$が成り立つ。
\begin{align}
    g'\qty(\pdv{x^\rho}{x^\mu}\pdv{x'^\rho},\pdv{x^\sigma}{x^\nu}\pdv{x'^\sigma}) &= \Lambda g\qty(\pdv{x^\mu},\pdv{x^\nu})\\
    g'_{\rho\sigma}\pdv{x^\rho}{x^\mu}\pdv{x^\sigma}{x^\nu} &= \Lambda g_{\mu\nu}
\end{align}
ここでは$M=M',g=g'$とすると次の関係が成り立つ。
\begin{align}
    \eta_{\rho \sigma}\pdv{x'^\rho}{x^\mu}\pdv{x'^\sigma}{x^\nu} &= \Lambda(x) \eta_{\mu\nu}\label{3}
\end{align}
次に線素を考える。線素は$ds^2 = \eta_{\mu\nu}dx^\mu dx^\nu$で与えられる。共形変換の下での線素は
\begin{align}
ds^2= \eta_{\rho \sigma}dx'^\rho dx'^\sigma
= \eta_{\rho \sigma}\pdv{x'^\rho}{x^\mu}\pdv{x'^\sigma}{x^\nu}dx^\mu dx^\nu
= \Lambda \eta_{\mu\nu}dx^\mu dx^\nu
\end{align}
つまり、共形変換とはローレンツ変換とスケール変換の組み合わせであることがわかる。
\subsection{共形群}
2つの共形変換の合成も共形変換であるので、共形変換全体は群を成す。この群を共形群と呼ぶ。$\Lambda(x)=1$ならポアンカレ群に対応する。
\subsection{無限小変換}
flat spaceにおける無限小共形変換を考える。$x^\mu \to x'^\mu = x^\mu + \epsilon^\mu(x) (\epsilon<<1)$とする。
これを(\ref{3})式に代入すると
\begin{align}
    \eta_{\rho \sigma}\pdv{x'^\rho}{x^\mu}\pdv{x'^\sigma}{x^\nu} &= \Lambda(x) \eta_{\mu\nu}\\
    \eta_{\rho \sigma}\qty(\delta^\rho_\mu + \pdv{\epsilon^\rho}{x^\mu})\qty(\delta^\sigma_\nu + \pdv{\epsilon^\sigma}{x^\nu}) &= \qty(1+K(x)) \eta_{\mu\nu}\\
    \partial_\mu \epsilon_\nu + \partial_\nu \epsilon_\mu &= K(x) \eta_{\mu\nu}\label{7}
\end{align}
この$K(x)$をどうにか表せないか考える。トレースをとる操作を行う。つまり両辺に$\eta^{\nu\mu}$
\begin{align}
    \partial_\mu \epsilon^\mu + \partial_\nu \epsilon^\nu &= K(x) \eta_{\mu\nu}\eta^{\nu\mu}\\
    K(x)=\frac{2\partial^\mu\epsilon_\mu}{d}\\
\end{align}
これを(\ref{7})式に代入すると
\begin{align}
    \partial_\mu \epsilon_\nu + \partial_\nu \epsilon_\mu &= \frac{2}{d}(\partial\cdot\epsilon)\eta_{\mu\nu}\label{10}
\end{align}
両辺を$\partial^\nu$を作用させると
\begin{align}
    \partial^\nu\partial_\mu \epsilon_\nu + \partial^2 \epsilon_\mu &= \frac{2}{d}\partial^\nu\eta_{\mu\nu}(\partial\cdot\epsilon)\\
    \partial_\mu(\partial\cdot\epsilon) + \square \epsilon_\mu &= \frac{2}{d}\partial_\mu(\partial\cdot\epsilon)\\
\end{align}
これの$\mu$と$\nu$を入れ替えたものを用意し、辺辺足しあげると
\begin{align}
    2\partial_\mu\partial_\nu(\partial\cdot\epsilon) + 2\square \epsilon_\mu &= \frac{4}{d}\partial_\mu\partial_\nu(\partial\cdot\epsilon)\\
    d \partial_\mu\partial_\nu(\partial\cdot\epsilon) + \square \eta_{\mu\nu}(\partial \cdot \epsilon) &= 2\partial_\mu\partial_\nu(\partial\cdot\epsilon)\\
    \eta_{\mu\nu}\square(\partial\cdot\epsilon) +(d-2)\partial_\mu\partial_\nu(\partial\cdot\epsilon)&=0\label{14}
\end{align}
$\eta^{\mu\nu}$をかけてトレースをとると
\begin{align}   
    \square(\partial\cdot\epsilon) +(d-2)\square(\partial\cdot\epsilon)&=0\\
    d\square(\partial\cdot\epsilon)&=0
\end{align}



\end{document}